% Options for packages loaded elsewhere
\PassOptionsToPackage{unicode}{hyperref}
\PassOptionsToPackage{hyphens}{url}
\PassOptionsToPackage{dvipsnames,svgnames,x11names}{xcolor}
%
\documentclass[
  letterpaper,
  DIV=11,
  numbers=noendperiod]{scrreprt}

\usepackage{amsmath,amssymb}
\usepackage{iftex}
\ifPDFTeX
  \usepackage[T1]{fontenc}
  \usepackage[utf8]{inputenc}
  \usepackage{textcomp} % provide euro and other symbols
\else % if luatex or xetex
  \usepackage{unicode-math}
  \defaultfontfeatures{Scale=MatchLowercase}
  \defaultfontfeatures[\rmfamily]{Ligatures=TeX,Scale=1}
\fi
\usepackage{lmodern}
\ifPDFTeX\else  
    % xetex/luatex font selection
\fi
% Use upquote if available, for straight quotes in verbatim environments
\IfFileExists{upquote.sty}{\usepackage{upquote}}{}
\IfFileExists{microtype.sty}{% use microtype if available
  \usepackage[]{microtype}
  \UseMicrotypeSet[protrusion]{basicmath} % disable protrusion for tt fonts
}{}
\makeatletter
\@ifundefined{KOMAClassName}{% if non-KOMA class
  \IfFileExists{parskip.sty}{%
    \usepackage{parskip}
  }{% else
    \setlength{\parindent}{0pt}
    \setlength{\parskip}{6pt plus 2pt minus 1pt}}
}{% if KOMA class
  \KOMAoptions{parskip=half}}
\makeatother
\usepackage{xcolor}
\usepackage{svg}
\setlength{\emergencystretch}{3em} % prevent overfull lines
\setcounter{secnumdepth}{5}
% Make \paragraph and \subparagraph free-standing
\ifx\paragraph\undefined\else
  \let\oldparagraph\paragraph
  \renewcommand{\paragraph}[1]{\oldparagraph{#1}\mbox{}}
\fi
\ifx\subparagraph\undefined\else
  \let\oldsubparagraph\subparagraph
  \renewcommand{\subparagraph}[1]{\oldsubparagraph{#1}\mbox{}}
\fi


\providecommand{\tightlist}{%
  \setlength{\itemsep}{0pt}\setlength{\parskip}{0pt}}\usepackage{longtable,booktabs,array}
\usepackage{calc} % for calculating minipage widths
% Correct order of tables after \paragraph or \subparagraph
\usepackage{etoolbox}
\makeatletter
\patchcmd\longtable{\par}{\if@noskipsec\mbox{}\fi\par}{}{}
\makeatother
% Allow footnotes in longtable head/foot
\IfFileExists{footnotehyper.sty}{\usepackage{footnotehyper}}{\usepackage{footnote}}
\makesavenoteenv{longtable}
\usepackage{graphicx}
\makeatletter
\def\maxwidth{\ifdim\Gin@nat@width>\linewidth\linewidth\else\Gin@nat@width\fi}
\def\maxheight{\ifdim\Gin@nat@height>\textheight\textheight\else\Gin@nat@height\fi}
\makeatother
% Scale images if necessary, so that they will not overflow the page
% margins by default, and it is still possible to overwrite the defaults
% using explicit options in \includegraphics[width, height, ...]{}
\setkeys{Gin}{width=\maxwidth,height=\maxheight,keepaspectratio}
% Set default figure placement to htbp
\makeatletter
\def\fps@figure{htbp}
\makeatother
\newlength{\cslhangindent}
\setlength{\cslhangindent}{1.5em}
\newlength{\csllabelwidth}
\setlength{\csllabelwidth}{3em}
\newlength{\cslentryspacingunit} % times entry-spacing
\setlength{\cslentryspacingunit}{\parskip}
\newenvironment{CSLReferences}[2] % #1 hanging-ident, #2 entry spacing
 {% don't indent paragraphs
  \setlength{\parindent}{0pt}
  % turn on hanging indent if param 1 is 1
  \ifodd #1
  \let\oldpar\par
  \def\par{\hangindent=\cslhangindent\oldpar}
  \fi
  % set entry spacing
  \setlength{\parskip}{#2\cslentryspacingunit}
 }%
 {}
\usepackage{calc}
\newcommand{\CSLBlock}[1]{#1\hfill\break}
\newcommand{\CSLLeftMargin}[1]{\parbox[t]{\csllabelwidth}{#1}}
\newcommand{\CSLRightInline}[1]{\parbox[t]{\linewidth - \csllabelwidth}{#1}\break}
\newcommand{\CSLIndent}[1]{\hspace{\cslhangindent}#1}

\KOMAoption{captions}{tableheading}
\makeatletter
\makeatother
\makeatletter
\@ifpackageloaded{bookmark}{}{\usepackage{bookmark}}
\makeatother
\makeatletter
\@ifpackageloaded{caption}{}{\usepackage{caption}}
\AtBeginDocument{%
\ifdefined\contentsname
  \renewcommand*\contentsname{Table of contents}
\else
  \newcommand\contentsname{Table of contents}
\fi
\ifdefined\listfigurename
  \renewcommand*\listfigurename{List of Figures}
\else
  \newcommand\listfigurename{List of Figures}
\fi
\ifdefined\listtablename
  \renewcommand*\listtablename{List of Tables}
\else
  \newcommand\listtablename{List of Tables}
\fi
\ifdefined\figurename
  \renewcommand*\figurename{Figure}
\else
  \newcommand\figurename{Figure}
\fi
\ifdefined\tablename
  \renewcommand*\tablename{Table}
\else
  \newcommand\tablename{Table}
\fi
}
\@ifpackageloaded{float}{}{\usepackage{float}}
\floatstyle{ruled}
\@ifundefined{c@chapter}{\newfloat{codelisting}{h}{lop}}{\newfloat{codelisting}{h}{lop}[chapter]}
\floatname{codelisting}{Listing}
\newcommand*\listoflistings{\listof{codelisting}{List of Listings}}
\makeatother
\makeatletter
\@ifpackageloaded{caption}{}{\usepackage{caption}}
\@ifpackageloaded{subcaption}{}{\usepackage{subcaption}}
\makeatother
\makeatletter
\@ifpackageloaded{tcolorbox}{}{\usepackage[skins,breakable]{tcolorbox}}
\makeatother
\makeatletter
\@ifundefined{shadecolor}{\definecolor{shadecolor}{rgb}{.97, .97, .97}}
\makeatother
\makeatletter
\makeatother
\makeatletter
\makeatother
\ifLuaTeX
  \usepackage{selnolig}  % disable illegal ligatures
\fi
\IfFileExists{bookmark.sty}{\usepackage{bookmark}}{\usepackage{hyperref}}
\IfFileExists{xurl.sty}{\usepackage{xurl}}{} % add URL line breaks if available
\urlstyle{same} % disable monospaced font for URLs
\hypersetup{
  pdftitle={Geospatial Data Science with Julia},
  pdfauthor={Júlio Hoffimann},
  colorlinks=true,
  linkcolor={blue},
  filecolor={Maroon},
  citecolor={Blue},
  urlcolor={Blue},
  pdfcreator={LaTeX via pandoc}}

\title{Geospatial Data Science with Julia}
\author{Júlio Hoffimann}
\date{2023-08-24}

\begin{document}
\maketitle
\ifdefined\Shaded\renewenvironment{Shaded}{\begin{tcolorbox}[borderline west={3pt}{0pt}{shadecolor}, enhanced, frame hidden, interior hidden, breakable, sharp corners, boxrule=0pt]}{\end{tcolorbox}}\fi

\renewcommand*\contentsname{Table of contents}
{
\hypersetup{linkcolor=}
\setcounter{tocdepth}{2}
\tableofcontents
}
\bookmarksetup{startatroot}

\hypertarget{welcome}{%
\chapter*{Welcome}\label{welcome}}
\addcontentsline{toc}{chapter}{Welcome}

\markboth{Welcome}{Welcome}

\emph{Geospatial Data Science with Julia} presents a fresh approach to
data science with geospatial data and the
\href{https://julialang.org}{\includesvg[width=0.41667in,height=\textheight]{index_files/mediabag/julia-logo-color.svg}}
programming language. It contains best practices for writting
\emph{clean}, \emph{readable} and \emph{performant} code in
geoscientific applications involving sophisticated representations of
the (sub)surface of the Earth such as unstructured meshes made of 2D and
3D geometries.

By reading this book, you will:

\begin{enumerate}
\def\labelenumi{\arabic{enumi}.}
\tightlist
\item
  Get a broader perspective on geospatial data
\item
  Learn advanced geostatistical algorithms
\item
  Reproduce practical \textbf{open source} examples
\end{enumerate}

Most importantly, you will learn a set of geospatial features that is
much richer than the
\href{https://en.wikipedia.org/wiki/Simple_Features}{simple features}
implemented in traditional geographic information systems (GIS).

This work is licensed under a Creative Commons
Attribution-NonCommercial-NoDerivatives 4.0 International License.

\hypertarget{how-to-contribute}{%
\section*{How to contribute?}\label{how-to-contribute}}
\addcontentsline{toc}{section}{How to contribute?}

\markright{How to contribute?}

First off, thank you for considering contributing to this book. It's
people like you that make open source projects so much fun. Below are a
few suggestions to facilitate the review process:

\begin{itemize}
\tightlist
\item
  Please be polite, we are here to help and learn from each other
\item
  Try to explain your contribution with simple language
\item
  References to textbooks and papers are always welcome
\item
  Follow the code style in the examples as much as possible
\end{itemize}

This book is
\href{https://github.com/juliohm/geospatial-data-science-with-julia}{open
source} and fully reproducible thanks to the amazing
\href{https://quarto.org}{Quarto} project. You can edit the pages
directly on GitHub and submit a pull request for review. If you are not
familiar with this process, consider reading the
\href{https://github.com/firstcontributions/first-contributions}{first
contributions} guide.

Alternatively, you can render the book locally with the
\href{https://marketplace.visualstudio.com/items?itemName=quarto.quarto}{Quarto
VS Code Extension}, which is the recommended method for reviewing more
elaborate changes.

\hypertarget{getting-involved}{%
\section*{Getting involved}\label{getting-involved}}
\addcontentsline{toc}{section}{Getting involved}

\markright{Getting involved}

If you would like to get involved with the project, you can start by:

\begin{itemize}
\tightlist
\item
  \href{}{Sharing} the book on social media
\item
  \href{https://github.com/juliohm/geospatial-data-science-with-julia}{Starring}
  the GitHub repository
\item
  \href{https://github.com/JuliaEarth/GeoStats.jl}{Starring} the
  accompanying software
\item
  \href{CITATION.bib}{Citing} the book in publications
\item
  \href{https://github.com/juliohm/geospatial-data-science-with-julia}{Asking}
  questions and making suggestions
\end{itemize}

\bookmarksetup{startatroot}

\hypertarget{foreword}{%
\chapter*{Foreword}\label{foreword}}
\addcontentsline{toc}{chapter}{Foreword}

\markboth{Foreword}{Foreword}

I've always felt that something was wrong with current approaches to
geospatial data science in other programming languages. I remember
sitting in the beautiful
\href{https://en.wikipedia.org/wiki/Cecil_H._Green_Library}{Stanford's
Green library} after a short introductory course on GIS with R, and
wondering why it had to be so ``computer sciency'':

\begin{itemize}
\tightlist
\item
  What on Earth is a ``LineString''? Isn't a ``Line'' a geometric object
  with infinite length?
\item
  Why do I need to learn this low-level distinction between ``raster''
  and ``vector'' data?
\end{itemize}

Something was off for sure, but it took me years after my PhD to
envision a more general approach to geospatial data science through the
continuous development of the
\href{https://github.com/JuliaEarth/GeoStats.jl}{GeoStats.jl} framework.
I have to say, it is not easy to swim against the wave of
\href{https://www.iso.org/standard/40114.html}{simple features}. Most
software developers out there are still surfing this
\textasciitilde20-year-old standard in spite of pretty major
limitations. Geoscience is \textbf{not} simple, is \textbf{not} 2D, and
should \textbf{not} be treated as a software engineering endeavour.

In that sense, I am one of the few rebels out there (hope you will join
us!) that devotes energy and time to educate geoscientists about this
new way of working with geospatial data. There is a long journey until
the technology reaches its full potential, but it is getting there! I
hope that the book will enlighten your understanding of some of those
issues and that you can benefit from all the \textbf{open source}
software we built over the years.

\bookmarksetup{startatroot}

\hypertarget{preface}{%
\chapter*{Preface}\label{preface}}
\addcontentsline{toc}{chapter}{Preface}

\markboth{Preface}{Preface}

\hypertarget{who-this-book-is-for}{%
\section*{Who this book is for}\label{who-this-book-is-for}}
\addcontentsline{toc}{section}{Who this book is for}

\markright{Who this book is for}

Anyone interested in \textbf{geospatial data science} will benefit from
reading this book. If you are a student with basic-to-intermediate
programming experience, you will learn a valuable set of tools for your
career. If you are an experienced data scientist, you can still be
surprised by the generality of the framework presented here.

This is \textbf{not} a book on geostatistics. Although some chapters and
examples will cover basic concepts from geostatistical theory, that is
only to illustrate what is possible after you master geospatial data
science with the
\href{https://julialang.org}{\includesvg[width=0.41667in,height=\textheight]{index_files/mediabag/julia-logo-color.svg}}
programming language.

\hypertarget{why-julia}{%
\section*{Why Julia?}\label{why-julia}}
\addcontentsline{toc}{section}{Why Julia?}

\markright{Why Julia?}

An effective implementation of the framework presented in this book
requires a language that can:

\begin{itemize}
\tightlist
\item
  Generate high-performance code
\item
  Specialize on multiple arguments
\item
  Evaluate code interactively
\item
  Exploit parallel hardware
\end{itemize}

This list of requirements eliminates Python, R and other mainstream
languages used for data science.

\hypertarget{how-to-read-this-book}{%
\section*{How to read this book}\label{how-to-read-this-book}}
\addcontentsline{toc}{section}{How to read this book}

\markright{How to read this book}

If this is your first encounter with
\href{https://julialang.org}{\includesvg[width=0.41667in,height=\textheight]{index_files/mediabag/julia-logo-color.svg}}
or with programming in general, consider reading the open source book
\href{https://benlauwens.github.io/ThinkJulia.jl/latest/book.html}{\emph{Think
Julia: How to Think Like a Computer Scientist}} by Lauwens, B. \&
Downey, A. 2018. It introduces the language to first-time programmers
and explains basic concepts that you will need to know to master the
material here.

If you are an experienced programmer who just wants to quickly learn the
syntax of the language, consider checking the
\href{https://learnxinyminutes.com/docs/julia}{Learn Julia in Y minutes}
website. If you are seeking more detailed information, consider reading
the \href{https://docs.julialang.org/en/v1}{official documentation}.

Assuming that you learned the basics of the language, you can proceed
and read this book. It is organized in three parts as follows:

\hypertarget{acknowledgements}{%
\section*{Acknowledgements}\label{acknowledgements}}
\addcontentsline{toc}{section}{Acknowledgements}

\markright{Acknowledgements}

\bookmarksetup{startatroot}

\hypertarget{introduction}{%
\chapter{Introduction}\label{introduction}}

This is a book created from markdown and executable code.

See Knuth (1984) for additional discussion of literate programming.

\bookmarksetup{startatroot}

\hypertarget{references}{%
\chapter*{References}\label{references}}
\addcontentsline{toc}{chapter}{References}

\markboth{References}{References}

\hypertarget{refs}{}
\begin{CSLReferences}{1}{0}
\leavevmode\vadjust pre{\hypertarget{ref-knuth84}{}}%
Knuth, Donald E. 1984. {``Literate Programming.''} \emph{Comput. J.} 27
(2): 97--111. \url{https://doi.org/10.1093/comjnl/27.2.97}.

\end{CSLReferences}



\end{document}
